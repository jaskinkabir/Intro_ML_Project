\documentclass[conference]{IEEEtran}
\IEEEoverridecommandlockouts
% The preceding line is only needed to identify funding in
% the first footnote. If that is unneeded, please comment it
% out.
\usepackage{cite}
\usepackage{hyperref}
\usepackage{amsmath,amssymb,amsfonts}
\usepackage{algorithmic}
\usepackage{graphicx}
\usepackage{textcomp}
\usepackage{xcolor}
\usepackage{placeins}
\usepackage{titlesec}

\setcounter{secnumdepth}{4}



\def\BibTeX{{\rm B\kern-.05em{\sc i\kern-.025em
    b}\kern-.08em
    T\kern-.1667em\lower.7ex\hbox{E}\kern-.125emX}}
\begin{document}

\title{ Fire Detection With Erroneous Input Correction\\
        { \footnotesize ECGR 4105/5105 - Introduction to
            Machine Learning} }

\author{\IEEEauthorblockN{1\textsuperscript{st} Jaskin
Kabir} \IEEEauthorblockA{\textit{ECGR Student} \\
\textit{UNC Charlotte}\\
Charlotte, NC \\
jkabir@charlotte.edu}
\and
\IEEEauthorblockN{2\textsuperscript{nd} Axel Leon Vasquez}
\IEEEauthorblockA{\textit{ECGR Student} \\
\textit{UNC Charlotte}\\
Charlotte, NC \\
aleonvas@charlotte.edu}
\and
\IEEEauthorblockN{3\textsuperscript{rd} Matthew Anderson}
\IEEEauthorblockA{\textit{ECGR Student} \\
\textit{UNC Charlotte}\\
Charlotte, NC \\
mande137@charlotte.edu} }
\maketitle

\begin{abstract}
The accuracy of typical fire alarm systems is often limited
due to their reliance on single sensors with constrained
detection capabilities. Moreover, a sensor failure renders
the entire system ineffective, posing a critical risk. While
AI-based fire detectors have massively outperformed
traditional systems, little work has addressed handling
sensor failures. To address this challenge, we propose
\textbf{AlarmNet}, a robust fire detection system that
incorporates erroneous sensor data during training.
We investigate two implementations of AlarmNet: a
centralized model trained on a single server and a
simulation of a federated model where training is
distributed across multiple clients and aggregated into a
global model. 
In our experiments, sensor errors were introduced
by randomly replacing 24.2\% of the dataset’s measurements
with missing or faulty values. We applied a pre-processing
step that replaced these values with the median of the
corresponding training features. Our 3-layer artificial
neural network The model achieved an accuracy of 98.38\% and
successfully detected 99.73\% of fires, despite the high
rate of sensor errors. This resulted in only a 0.25\%
reduction in detection performance compared to a model
trained on complete data, and a significant 41.51\%
improvement over a simulated commercial-grade fire alarm
system." The federated approach delivered comparable results
while utilizing a less computationally intensive model.
In future work, we aim to extend this approach to develop a
real-world fire detection system capable of maintaining
robustness against sensor failures.
\end{abstract}

\begin{IEEEkeywords}
machine learning (ML), algorithm, model. Artificial Neural
Network, Imputation, Sensor fusion, Fire Alarms, Fire
Detection, Error Handling, Federated Learning
\end{IEEEkeywords}

\section{Introduction and Motivation}
\subsection{Introduction}\label{intro} 

Fire alarm systems play a critical role in safeguarding
lives and property by providing early warnings of fire
hazards. Traditional systems, however, often rely on single
sensors with limited detection capabilities. These systems
are highly susceptible to sensor failures, which can render
them ineffective and pose significant safety risks. In
addition, the prevalence of false alarms reduces their
reliability and can lead to complacency or unnecessary
emergency responses. While artificial intelligence
(AI)-based fire detectors have demonstrated significant
improvements in accuracy and performance over traditional
systems, they generally assume clean, error-free sensor
data. This assumption limits their practical applicability,
especially in real-world scenarios where sensor errors and
failures are common.

To address these challenges, we propose AlarmNet, a robust
AI-based fire detection system designed to handle sensor
errors effectively. AlarmNet is trained on a dataset from which
randomly selecteded measurements were removed. This approach
enables the model to learn to detect fires accurately even
in the presence of sensor errors. We explore
two implementations of AlarmNet: a centralized model trained
on a single server and a federated model where training is
distributed across multiple clients and aggregated into a
global model. This dual approach ensures scalability and
flexibility while maintaining high detection accuracy. In
this paper, we detail the design, implementation, and
evaluation of AlarmNet, demonstrating its effectiveness in
handling real-world sensor failures. A 2013 study found the
Honeywell FS90, a widely used commercial fire detection
system, only achieves an accuracy of 87.5\%\cite{smokeacc}.
This contributes to deaths, injuries, and damage due to
fires in two ways: Firstly when a fire occurs and the fire
detection system fails to recognize it, the occupants of the
building can only react to the fire once they are close
enough observe it, which often means it is too late to
evacuate. Secondly, a 1995 report stated that fewer than
25\% of residents of mid-rise residential buildings
interpreted the sound of the fire alarm as a potential
indication of a real emergency\cite{crywolf}. As false--or
'nuisance'--alarms are so common, even when the alarm sounds
for a real fire, occupants may not evacuate since they won't
take it seriously.

This performance suggests room for improvment. Machine
learning-based fire alarms have vastly improved fire alarm
accuracy, reducing false positives and enabling more
reliable detection.

\subsection{Motivation}\label{mot}

mot

\section{Approach}

\section{Dataset and Training Setup}
\subsection{Dataset}
The dataset used in this project is the Smoke Detection
Dataset provided by Stefan Blattmann. The dataset consists
of approximately 62,629 sensor readings encompassing both
normal and fire-related scenarios, providing diverse
training and evaluating fire detection models.

\subsubsection{Data Preparation}


\begin{itemize}
\item Data Features: Environmental features such as
temperature, humidity, and pressure, with particulate matter
readings such as PM1.0, PM2.5, NC0.5, NC1.0, and NC2.5
measuring air quality sensors. 
\item Gas Sensor Outputs: TVOC ( Total Volatile Organic
Compounds)  indicates the presence of volatile organic
compounds in combustion. eCO2 contains high CO2
concentration often indicating fire activities. Raw H2 and
Raw Ethanol are outputs from chemical sensors detecting
hydrogen and ethanol which are released during certain types
of fires.  
\item Handling Sensor Errors: Simulates sensor malfunctions
to evaluate error-handling capabilities. Including
thresholds to identify invalid readings and imputed missing
values within the median or mean imputation. 
\item Correlation: Analysis to rank features by importance
reduction from 12 features to 4 key features, humidity, Raw
Ethanol, Pressure, and TVOC. Additionally, PM0.5 for sensor
redundancy, resulting in a 5-feature model.

\subsubsection{Feature Selection}
The smoke detection system was crucial in developing an
efficient and robust model. Starting with 15 initial
features, only 12 were unable features, after removing
non-informative columns such as UTC Timestamps, CNT, and
Unnamed :0. Optimizing the model's performance and reducing
computational complexity, tests were conducted such as
correlation analysis, ranking features based on their
absolute correlation with 'Fire Alarm" target variables.
This test led to a significant reduction in the feature set
in a compact yet highly effective 5-feature model that
maintained excellent performance within sensor redundancy
and avoided overfitting. 
\item Correlation Analysis Methodology: it utilized absolute
correlation values with 'Fire Alarm' as the target variable.
It uses ranking features based on the strength to get rid of
the non-use features, such as TVOC, PM2.5, and NC1.0 show
strong correlations with the presence of fire. 
\item Dimensionality Reduction: The feature set from 12 to 4
key features significantly lowers model complexity. It also
demonstrated the trade-off between performance and
simplicity, by removing 8 features with minimal impact. 
\item Sensor Redundancy: Adding the PM0.5 as the fifth
feature represented all four sensors which explains the
importance of maintaining sensor diversity for robustness
thus approaching the creation of a balance model capturing
key patterns while avoiding overfitting. 
\end{itemize}

\subsection{Error Insertion}

\subsection{Model Architecture}
The model implemented a custom deep neural network
(AlarmNet) for fire detection. The dataset's multivariate
sensor data balance complexity and efficiency. Input layer
size depends on the number of features such as 12 features
as the initial full-feature model. It was reduced to 4
features via a correlation ranking system. Hidden layers
were added with (62,32) for a 12-feature model and a
5-feature model. The 4-feature system uses (64,64) to
provide additional capacity. All these models activated
functions with ReLU( Rectified Linear Unit) introducing
non-linearity and mitigating the vanishing gradient problem.
The training model algorithm implemented the Adam Optimizer
technique for its adaptive learning rate abilities with an
initial learning rate of 1e-3 for most models and 1e-5 for
the imputed error model. Parameters are updated using
gradient descent with momentum adjustment for faster
convergence. The learning rate is based on validation loss
improvements, configured with a factor of 0.5 to halve the
learning rate when performed. Having a patience of 4000
iterations without improvement and a threshold of 1e-3 for
significant improvements. In the training configuration,
3000 epoch models were extended to 16,000 for the imputed
error model to ensure convergence. Using epoch will improve
based on the device GPU accelerated training on CUDA when
available, otherwise CPU. 




\subsection{Evaluation Methods}
\subsubsection{Metrics}
\begin{itemize}
\item Data Splitting: The dataset is split into 80/20
percent for training/testing to evaluate model performance. 
\item Hardware: A code was implemented to perform on
GPU(CUDA) for acceleration; otherwise CPU is available. 
\item Training Configuration: Adam optimizer was implemented
to configure with learning rates of 1e-3 for most models and
1e-5 for the imputed error model. Learning rate by a factor
of 0.5, the patience of 4000 epochs, and threshold 1e-3. In
most models, 3000 epochs were implemented while 16,000
epochs were for the imputed error model to ensure proper
convergence on data. 
\item Precision: Proportion of correctly identified fire
alarms 
\begin{equation}
    \text{Precision} = \frac{\text{TP}}{\text{TP} + \text{FP}}
\end{equation}
\item Recall: Measures the proportion of actual fire cases
detected by the model to minimize false negatives
\begin{equation}
    \text{Recall} = \frac{\text{TP}}{\text{TP} + \text{FN}}
\end{equation}
\item F1-Score and Accuracy: Implements harmonic mean of
precision and recall with overall correctness of the model.
\begin{equation}
    F1 = 2 \times \frac{\text{Precision} \times \text{Recall}}{\text{Precision} + \text{Recall}}
\end{equation}
\item FPR: False Positive Rate measures the actual negatives
that are incorrectly classified as positives 
\begin{equation}
    \text{FPR} = \frac{\text{FP}}{\text{FP} + \text{TN}}
\end{equation}
\item Confusion Matrix: This model visualizes true
positive/negative and false positive/negatives
\end{itemize}
\subsubsection{Honeywell FS90 Simulation}



\subsection{Model Architecture and Training}

\textbf 
In the real world sensor data, is often due to hardware
communication limitations and errors. It can lead to
incomplete or noisy environmental factors. The reliability
of the model can go in such conditions in error handling
using robust strategies to incorporate into the model and
training process. These strategies allowed the model to
maintain high accuracy and recall during the sensor error
process. In simulating real-world scenarios, errors were
intentionally introduced into the dataset. In the model run,
20 percent was applied across all features, with particulate
matter (PM) sensors likely to encounter errors due to
insufficient redundancy. The dataset showed four sensors in
their categories, temperature and Humidity, Pressure, (TVOC,
CO2, Ethanol, H2), and Particulate Matter ( NC0.5, NC1.0,
NC2.5, PM1.0, PM2.5). erroneous values were represented as
missing data (NaN) or extreme outliers to emulate hardware
failures or any communication drops. Missing values were
handled using median imputation, which replaces erroneous
values with the median of the corresponding feature
calculated from the training dataset. The median was chosen
against outliers suitable for skew distributions in sensor
data. Imputation implementation ensures the dataset remains
intact, allowing it to learn effectively. The feature was
further improved by identifying the most correlated features
with fire detection. Key features such as TVOC, Pressure,
Raw Ethanol, and Humidity were retained.    
The training process specialized adjustments for handling
noisy and incomplete data. During the deep ANN error
handling architecture of [(256,256,256)] for better
resolution. Extending training epochs were used, enabling
the model to generalize effectively. In the error handling
stage, errors were introduced to replicate real-world
conditions challenging the robustness of the architecture.
The implementation strategy introduced erroneous or missing
values were replaced with the median of each feature to
maintain data integrity. The imputation approach was
particularly effective for skewed feature distributions
which enchanted the model ([256,256,256]) and demonstrated
resilience to noisy and incomplete data. The imputed error
model showcased its ability to handle a 20 percent sensor
error rate with minimal performance degradation and
showcasing minimized false negatives under challenging
conditions. Performance metrics were involved in all models
to evaluate using recall, precision, F1-Score, and confusion
matrices. The model was benchmarked against the Honeywell
FS90 fire detection system. Comparison revealed superior
recall and resilience in the proposed model, under simulated
noisy conditions, such as histograms of missing values and
bar charts comparing recall and recall without errors
showing the model's robustness. Furthermore, the size and
fast inference of the final model exported in ONNX format
ensure the deployment in real-time IoT scenarios of fire
detection systems. These error-handling strategies
demonstrate the model's ability to adapt to real-world
challenges for a reliable solution in fire detection in
different environments. Simulating error handling,
imputation, and targeted features in correlation highlights
the model's resilience and practical applications. 

\section{Results and Analysis}
In a comparative analysis, SVM(Support Vector Machine) and
NN (Neural Network) were both considered for fire detection
within the 12-feature test bench. Neural networks are
ultimately chosen for their superior flexibility and
robustness. SVMs are known for their simplicity and
effectiveness within smaller datasets due to their ability
to model linear and non-linear relationships using kernel
functions. However, in this dataset, the project had a large
size of 62,000 samples which limited the SVM training model
as memory requirement scale quadratically with the number of
samples. On the other hand neural network demonstrates
superior performance and adaptability. They efficiently
scaled with large datasets by leveraging batch size gradient
descent and GPU acceleration with mixed precision and
gradient scaling. It can learn hierarchical features
representing them as particularity effective for
multi-sensor data its architecture model is tailored to
handle noisy and missing data, as seen in the imputed error
model under results., It can maintain high recall and
precision under a 20 percent error rate. The full feature
model utilizes all 12 -features that achieved high precision
and recall but its computational complexity made it less
efficient. After using a correlation technique ranking them
by target variable it reduced to 4 and 5 feature sets
retaining only the top four features ( Humidity, Raw
Ethanol, Pressure, and TVOC), achieved performance to the
full feature model with minimal precision loss of 0.1
percent. It significantly improved making it more suitable
for different environments. The redundant feature model
added the NC0.5 enchanting redundancy and improved
robustness maintaining efficiency. The imputed error model
designed to handle noisy and missing data shows a resilience
with 20 percent error rate. Thus outperforming the Honeywell
FS90 system. In addition to accuracy and recall, time was
evaluated to determine real-time applicability. The proposed
models achieve an average time of 2e-2 seconds per sample
making them viable for deployment in fire detection systems.
Evaluation metrics showed visual comparison including
confusion matrices, and bar charts illustrating the superior
performance. The reduced features and imputed error and
aggregation demonstrated the ability to balance accuracy,
efficiency, and robustness for real-world fire detection
applications. The Federated model trains the local copy of a
global model using local data. The training sends model
weights (weights of the neurons in the neural network) to
the global mode. The federate and AlarmNet achieve near
identical accuracy. These bar graphs show the comparison of
recall with and without error and precision with 20 percent
error and without. With Machine Learning algorithms, it was
able to provide excellent validation against Honeywell FS90
highlighting the potential ANN that can provide for
enchanting better real-world safety protocols (fig 18 and
19). 

\section{Lessons Learned}
This machine learning model successfully demonstrated the
development an efficient fire detection system using sensor
data. By leveraging a custom neural network architecture,
the system achieved high precision and recall even with
real-world scenarios such as noisy or missing sensor data.
Feature selections and target preparation, including median
imputation and error handling, ensure the model's
exploration. The reduced feature model maintained comparable
performance to the 12-feature model significantly reducing
computational requirements and making it good for
constrained environments. Comparative analysis highlights
the proposed neural network over the traditional such as
SVM. The neural network can handle large datasets, capture
complex features tolerate sensor errors underscores the
effects in IoT-base fire detection systems. Optimizing time
and compact model size ensure viability for read-time
devices. The project object results emphasize the importance
of designing models that are balanced, efficient, accurate,
and robust for safety and critical protocol applications.
Future work can be explored further optimizing the learning,
advancing imputation, and improving the federated model
weights between three clients to enhance system performance.
The project demonstrates the potential for machine learning
to revolutionize fire detection technology to challenge
high-end companies such as Honeywell to ensure the power of
ANN to contribute to improving safety and reliability in
diverse environments.






\section{Contributions}
\begin{itemize}
\item\textbf{Matthew Anderson:} 
\begin{itemize}
    \item Federated Learning Class
    \item Edited Report and Presentation
\end{itemize}

\item\textbf{Jaskin Kabir:} 
\begin{itemize}
    \item Project Manager
    \item Dataset Selection, Preparation, and Analysis
    \item AlarmNet Class that contained the fully connected
    model
    \item Error insertion and imputation
    \item Edited Report and Presentation
\end{itemize}

\item\textbf{Axel Leon Vasquez:} 
\begin{itemize}
    \item Wrote Report
    \item Created Presentation
    \item Data Visualization
\end{itemize}
\end{itemize}
\subsection{References}

\bibliographystyle{IEEEtran}
\bibliography{refs.bib}



\subsection{Results} 

\begin{figure}
    \centering
    \includegraphics[width=0.75\linewidth]{images/SVM.png}
    \caption{SVM Metric}
    \label{fig: SVM Model}
\end{figure}

\begin{figure}
    \centering
    \includegraphics[width=0.75\linewidth]{images/12acc.png}
    \caption{12 - Feature Metrics}
    \label{fig: 1.0 }
\end{figure}

\begin{figure}
    \centering
    \includegraphics[width=0.75\linewidth]{images/12CM.png}
    \caption{12 - Feature Loss Curve}
    \label{fig: 1.2}
\end{figure}

\begin{figure}
    \centering
    \includegraphics[width=0.75\linewidth]{images/12CMM.png}
    \caption{12 - Feature Confusion Matrix}
    \label{fig:1.3}
\end{figure}

\begin{figure}
    \centering
    \includegraphics[width=0.7\linewidth]{images/Corr.png}
    \caption{Features}
    \label{fig:2.0-label}
\end{figure}

\begin{figure}
    \centering
    \includegraphics[width=0.75\linewidth]{images/4metric.png}
    \caption{4 - Feature Metrics}
    \label{fig:3.0}
\end{figure}

\begin{figure}
    \centering
    \includegraphics[width=0.75\linewidth]{images/4LC.png}
    \caption{4 - Feature Loss Curves}
    \label{fig:3.1}
\end{figure}

\begin{figure}
    \centering
    \includegraphics[width=0.75\linewidth]{images/4CM.png}
    \caption{4 - Feature Confusion Matrix}
    \label{fig:3.2}
\end{figure}

\begin{figure}
    \centering
    \includegraphics[width=0.75\linewidth]{images/5metric.png}
    \caption{5 - Feature Metrics}
    \label{fig:4.0}
\end{figure}

\begin{figure}
    \centering
    \includegraphics[width=0.75\linewidth]{images/5LC.png}
    \caption{5 - Feature Loss Curves}
    \label{fig:4.1}
\end{figure}

\begin{figure}
    \centering
    \includegraphics[width=0.75\linewidth]{images/5CM.png}
    \caption{5 - Feature Confusion Matrix}
    \label{fig:4.2}
\end{figure}

\begin{figure}
    \centering
    \includegraphics[width=0.75\linewidth]{images/imputationmetric.png}
    \caption{Imputation Metrics}
    \label{fig:5.0}
\end{figure}
\begin{figure}
    \centering
    \includegraphics[width=0.75\linewidth]{images/ImputationLC.png}
    \caption{Imputation Loss Curves}
    \label{fig:5.1}
\end{figure}

\begin{figure}
    \centering
    \includegraphics[width=0.75\linewidth]{images/ImputationCM.png}
    \caption{Imputation Confusion Matrix}
    \label{fig:5.2}
\end{figure}


\begin{figure}
    \centering
    \includegraphics[width=1\linewidth]{images/12Fed.png}
    \caption{Federarted Learning: 12-Features}
    \label{fig:6.0}
\end{figure}

\begin{figure}
    \centering
    \includegraphics[width=1\linewidth]{images/5Fed.png}
    \caption{Federated Learning: 5-Features}
    \label{fig:6.1}
\end{figure}

\begin{figure}
    \centering
    \includegraphics[width=1\linewidth]{images/FedHandling.png}
    \caption{Federated Learning: Error handling}
    \label{fig:6.2}
\end{figure}

\begin{figure}
    \centering
    \includegraphics[width=0.75\linewidth]{images/Recall.png}
    \caption{Recall Comparison Honeywell FS90 Vs AlarmNet ANN Vs Federated}
    \label{fig:7.0}
\end{figure}

\begin{figure}
    \centering
    \includegraphics[width=0.75\linewidth]{images/PrecisionComparison.png}
    \caption{Precison Comparison Honeywell FS90 Vs AlarmNet ANN Vs Federated}
    \label{fig:7.1}
\end{figure}

\end{document}